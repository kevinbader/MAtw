\documentclass[thesis.tex]{subfiles}
\begin{document}
\selectlanguage{USenglish}
\chapter{Conclusions}
\label{ch:Conclusions}

In the course of this work, we have designed three different memetic algorithms. Their performance has been evaluated by comparing them to each other, as well as to the state-of-the-art algorithms \glsname{BBtw}~\parencite{bachoore-bodlaender-2006-branchAndBound}, \glsname{QuickBB}~\parencite{gogate-2004-quickbb}, \glsname{TabuTW}~\parencite{clautiaux-2004-tabu}, \glsname{ACSILS}~\parencite{hammerl-thesis,hammerl-paper}, \glsname{GA-tw}~\parencite{schafhauser-thesis,schafhauser-paper}, and \glsname{IHA}~\parencite{musliu-2008-ILS}.

Our algorithms have been optimized by parameter tuning software. To ensure that the results represent real-world performance, we have used separate sets of instances for training/tuning (79 instances) and for validation (27 instances). In order for the results to be meaningful, statistical significance tests have been employed for ranking the algorithms.

The proposed memetic algorithms differ with respect to solution quality, i.e., the width of the resulting tree decomposition, where \gls{MA3} takes the lead in almost all instances. They also differ substantially regarding their time response, with \gls{MA3} being quicker than \gls{MA2}, which is in turn quicker than \gls{MA1}. So overall, \gls{MA3} turned out to be superior to the more complex variants \gls{MA1} and \gls{MA2}.

In general, our memetic algorithms are competitive, but they do not dominate all state-of-the-art solvers for this problem. Nevertheless, \gls{MA3} has been able to find new best known upper bounds on 8 benchmark instances of the Second \gls{DIMACS} Implementation Challenge. One of them is also found by \gls{MA2}.

The results suggest that memetic algorithms are promising and worth investigating. Future work includes experiments using parallel implementations of \glspl{MA}, where a natural distinction between the phases of a memetic algorithm can help to optimize for concurrency. Since the way of combining two heuristics into a \gls{MA} has such a large impact on the performance, it might be worth investigating more variants, while aiming at an understanding of the underlying principles.

%Moscato~\parencite{moscato-1989} has also proposed several augmentations to his \gls{MA} design. The most highlighted addition is the replacement of the deterministic acceptance of competitions and cooperations with a stochastic strategy, using a dependency on a value that lessens over time. He denotes the corresponding variable \emph{temperature}, in reference to \emph{simulated annealing}. Other suggested methods include adaptive strategies, reheating, or restarting parts of the population, possibly guided by a population diversity metric. These augmentations could also help to further improve the performance of memetic algorithms.


\end{document}
% vim: set ts=3 sts=3 sw=3 tw=0 et :
